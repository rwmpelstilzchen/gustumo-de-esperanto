\bichapter{קישורים}{Ligiloj}

\begin{compactitem}
\item דקדוק:
\begin{compactitem}
\item תקציר לא רע בכלל של הדקדוק יש בוויקפדיה האנגלית: \url{en.wikipedia.org/wiki/Esperanto_grammar}
\item דקדוקים מפורטים, באספרנטו:
	\begin{compactitem}
	\item \EO{Plena Manlibro de Esperanta Gramatiko}: \url{bertilow.com/pmeg}
	\item \EO{Plena Analiza Gramatiko de Esperanto}: \url{eo.wikipedia.org/wiki/Plena_Analiza_Gramatiko}
	\end{compactitem}
\item „\EO{lernu!}” הוא אתר רב־לשוני ללימוד אספרנטו; הגרסה בעברית: \url{he.lernu.net}
\end{compactitem}

\item מילון:
\begin{compactitem}
\item מידע מילונאי מוויקיפדיה האנגלית:
	\begin{compactitem}
	\item על אוצר־המילים: \url{en.wikipedia.org/wiki/Esperanto_vocabulary}
	\item על מקורות המילים: \url{en.wikipedia.org/wiki/Esperanto_etymology}
	\end{compactitem}
\item על מבנה המילים: \url{cindymckee.com/librejo/Word_formation_in_Esperanto.pdf}
\item מילונים:
	\begin{compactitem}
	\item באתר „\EO{lernu!}” מילונים דו־כיווניים: \url{he.lernu.net}. שימו לב: המילון אספרנטו{\fontspec{Symbola}↔}עברית פחות מקיף מאספרנטו{\fontspec{Symbola}↔}אנגלית.
	\item עם התוכנה \EO{Esperantilo} מגיע מילונים שניתנים להורדה ולחיפוש בפורמט טקסט פשוט: \url{esperantilo.org}
	\item \EO{Reta Vortaro}, מילון אספרנטו-אספרנטו ורב־לשוני: \url{reta-vortaro.de}
	\item \EO{Plena Ilustrita Vortaro de Esperanto}, מילון אספרנטו-אספרנטו מקיף: \url{vortaro.net}
	\end{compactitem}
\end{compactitem}

\item ספרות:
\begin{compactitem}
\item ספרות מקור באספרנטו (סקירה, ביקורות, ובחלק מהיצירות גם גרסאות לקריאה במחשב): \url{esperanto.net/literaturo}
\item ספריה ממוחשבת, בעיקר של ספרות מתורגמת ישנה: \url{i-espero.info/elsutaro/esperantaj-libroj}
\item \EO{La Katalogo de UEA}, קטלוג של ספרות באספרנטו, כולל אפשרות להזמנה בדואר (נכון לזמן כתיבת שורות אלה, בקטלוג 7177 כותרים, מאת 3190 מחברים): \url{katalogo.uea.org}
\item תרגום עברי של סיפור באספרנטו, לצד המקור, ניתן למצוא בגליון הראשון של כתב־העת „בבל”: \url{bbl.digitalwords.net}
\end{compactitem}

\item חברה:
\begin{compactitem}
\item \EO{Pasporta Servo}, להתארח אצל ולארח אספרנטיסטים מכל העולם: \url{pasportaservo.org}
\item \EO{Koresponda Servo Universala}, התכתבות (בדואל או במכתבים וגלויות) עם אנשים מכל העולם: \url{esperanto-plus.ru/koresponda-servo}
\item כנסים:
	\begin{compactitem}
	\item \EO{Universala Kongreso de Esperanto}, הקונגרס העולמי לאספרנטו (הקונגרס הגדול ביותר; ב־2014 יתקיים הקונגרס ה־99, בבואנוס איירס. משנת 1905 הקונגרס מתקיים בכל שנה, עם שני „חורים” בזמן מלחמות העולם): \url{uea.org/kongresoj}
	\item \EO{Internacia Junulara Kongreso}, קונגרס הצעירים הבינלאומי (ב־2014 יתקיים הקונגרס ה־70, בפורטלזה): \url{tejo.org/ijk}
	\item \EO{Internacia Infana Kongreseto}, קונגרסון לילדים: \url{uea.org/kongresoj/iik.html}
	\item \EO{Azia Kongreso de Esperanto}, הקונגרס האסיאתי (ב־2014 יתקיים הקונגרס השמיני): \url{eo.wikipedia.org/wiki/Azia_Kongreso_de_Esperanto}
	\item \EO{La Mezorientaj Kunvenoj}, כנסים מזרח־תיכוניים (ב־2014 יתקיים הכנס השביעי, בטביליסי): \url{uea.org/vikio/La_Mezorientaj_Kunvenoj}
	\item \EO{Israela Kongreso de Esperanto}, קונגרס האספרנטו הישראלי (ב־2014 יתקיים הקונגרס ה־15): \url{esperanto.org.il/ik.html}
	\item בתל־אביב מתקיימים מפגשים שבועיים, ובחיפה ובירושלים~— חודשיים: \url{esperanto.org.il}
	\end{compactitem}
\item ארגונים:
	\begin{compactitem}
	\item \EO{Sennacieca Asocio Tutmonda}, האגודה האל־לאומית הכלל־עולמית: \url{satesperanto.org}
	\item \EO{Universala Esperanto-Asocio}, אגודת האספרנטו העולמית: \url{uea.org}
	\item \EO{Tutmonda Esperantista Junulara Organizo}, ארגון האספרנטיסטים הצעירים הכלל־עולמי: \url{tejo.org}
	\item \EO{Esperanto-Ligo en Israelo}, האגודה לאספרנטו בישראל מארגנת מפגשים, כנסים, קורסים, הוצאת בטאון חצי־שנתי ועוד: \url{esperanto.org.il}
	\end{compactitem}
\item פורום בעברית על אספרנטו: \url{tapuz.co.il/Forums2008/ForumPage.aspx?ForumId=181}
\end{compactitem}
\end{compactitem}
