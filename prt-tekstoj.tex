\bichapter{טקסטים}{Tekstoj}
\bisection{מגדל בבל}{Babela Turo}

\newcommand{\verso}[3]{\cxirkauxi{#1} & \BH{#2} & \EO{#3} & \cxirkauxi{#1}\\}
\begin{tabulary}{\textwidth}{CRLC}
	\verso{1}{וַֽיְהִ֥י כָל־הָאָ֖רֶץ שָׂפָ֣ה אֶחָ֑ת וּדְבָרִ֖ים אֲחָדִֽים׃}{Sur la tuta tero estis unu lingvo kaj unu parolmaniero.}
	\verso{2}{וַֽיְהִ֖י בְּנָסְעָ֣ם מִקֶּ֑דֶם וַֽיִּמְצְא֥וּ בִקְעָ֛ה בְּאֶ֥רֶץ שִׁנְעָ֖ר וַיֵּ֥שְׁבוּ שָֽׁם׃}{Kaj kiam ili ekiris de la oriento, ili trovis valon en la lando Ŝinar kaj tie ekloĝis.}
	\verso{3}{וַיֹּאמְר֞וּ אִ֣ישׁ אֶל־רֵעֵ֗הוּ הָ֚בָה נִלְבְּנָ֣ה לְבֵנִ֔ים וְנִשְׂרְפָ֖ה לִשְׂרֵפָ֑ה וַתְּהִ֨י לָהֶ֤ם הַלְּבֵנָה֙ לְאָ֔בֶן וְהַ֣חֵמָ֔ר הָיָ֥ה לָהֶ֖ם לַחֹֽמֶר׃}{Kaj ili diris unu al alia: Venu, ni faru brikojn kaj ni brulpretigu ilin per fajro. Kaj la brikoj fariĝis por ili ŝtonoj, kaj la bitumo fariĝis por ili kalko.} 
	\verso{4}{וַיֹּאמְר֞וּ הָ֣בָה ׀ נִבְנֶה־לָּ֣נוּ עִ֗יר וּמִגְדָּל֙ וְרֹאשׁ֣וֹ בַשָּׁמַ֔יִם וְנַֽעֲשֶׂה־לָּ֖נוּ שֵׁ֑ם פֶּן־נָפ֖וּץ עַל־פְּנֵ֥י כָל־הָאָֽרֶץ׃}{Kaj ili diris: Venu, ni konstruu al ni urbon, kaj turon, kies supro atingos la ĉielon, kaj ni akiru al ni gloron, antaŭ ol ni disiĝos sur la supraĵo de la tuta tero.} 
	\verso{5}{וַיֵּ֣רֶד יְהוָ֔ה לִרְאֹ֥ת אֶת־הָעִ֖יר וְאֶת־הַמִּגְדָּ֑ל אֲשֶׁ֥ר בָּנ֖וּ בְּנֵ֥י הָאָדָֽם׃}{Kaj la Eternulo malleviĝis, por vidi la urbon kaj la turon, kiujn konstruis la homidoj.} 
	\verso{6}{וַיֹּ֣אמֶר יְהוָ֗ה הֵ֣ן עַ֤ם אֶחָד֙ וְשָׂפָ֤ה אַחַת֙ לְכֻלָּ֔ם וְזֶ֖ה הַחִלָּ֣ם לַעֲשׂ֑וֹת וְעַתָּה֙ לֹֽא־יִבָּצֵ֣ר מֵהֶ֔ם כֹּ֛ל אֲשֶׁ֥ר יָזְמ֖וּ לַֽעֲשֽׂוֹת׃}{Kaj la Eternulo diris: Jen estas unu popolo, kaj unu lingvon ili ĉiuj havas; kaj jen, kion ili komencis fari, kaj ili ne estos malhelpataj en ĉio, kion ili decidis fari.} 
	\verso{7}{הָ֚בָה נֵֽרְדָ֔ה וְנָבְלָ֥ה שָׁ֖ם שְׂפָתָ֑ם אֲשֶׁר֙ לֹ֣א יִשְׁמְע֔וּ אִ֖ישׁ שְׂפַ֥ת רֵעֵֽהוּ׃}{Ni malleviĝu do, kaj Ni konfuzu tie ilian lingvon, por ke unu ne komprenu la parolon de alia.} 
	\verso{8}{וַיָּ֨פֶץ יְהוָ֥ה אֹתָ֛ם מִשָּׁ֖ם עַל־פְּנֵ֣י כָל־הָאָ֑רֶץ וַֽיַּחְדְּל֖וּ לִבְנֹ֥ת הָעִֽיר׃}{Kaj la Eternulo disigis ilin de tie sur la supraĵon de la tuta tero, kaj ili ĉesis konstrui la urbon.} 
	\verso{9}{עַל־כֵּ֞ן קָרָ֤א שְׁמָהּ֙ בָּבֶ֔ל כִּי־שָׁ֛ם בָּלַ֥ל יְהוָ֖ה שְׂפַ֣ת כָּל־הָאָ֑רֶץ וּמִשָּׁם֙ הֱפִיצָ֣ם יְהוָ֔ה עַל־פְּנֵ֖י כָּל־הָאָֽרֶץ׃}{Tial oni donis al ĝi la nomon Babel, ĉar tie la Eternulo konfuzis la lingvon de la tuta tero kaj de tie la Eternulo disigis ilin sur la supraĵon de la tuta tero.}
\end{tabulary}

\begin{LTR}
	\begin{enumerate}[label=\protect\cxirkauxi{\arabic*}]
		\item
			\gl{sur}{על}
			\gl{la}{ה־}
			\gl{ter·o}{ארץ, אדמה}
			\gl{est·i}{להיות}
			\gl{unu}{אחד}
			\gl{lingv·o}{שפה}
			\gl{kaj}{ו־}
			\gl{parol·manier·o}{דרך־דיבור}

		\item
			\gl{ki·am}{כש־, מתי}
			\gl{ili}{הם}
			\gl{ek·ir·i}{להתחיל ללכת}
			\gl{de}{של; מ־}
			\gl{orient·o}{מזרח}
			\gl{trov·i}{למצוא}
			\gl{val·o}{עמק, בקעה}
			\gl{en}{ב־; ל־ (עם \EO{-n})}
			\gl{land·o}{ארץ}
			\gl{Ŝinar}{שִׁנְעָר}
			\gl{ti·e}{שָׂם}
			\gl{ek·loĝ·i}{להתיישב, להתחיל לגור}

		\item
			\gl{dir·i}{לומר, להגיד}
			\gl{al}{אל, ל־}
			\gl{ali·a}{אַחֵר}
			\gl{ven·i}{לבוא}
			\gl{ni}{אנחנו}
			\gl{far·i}{לעשות}
			\gl{brik·o}{לבנה}
			\gl{brul·pret·ig·i}{לשרוף~+ להכין}
			\gl{per}{על־ידי}
			\gl{fajr·o}{אש}
			\gl{far·iĝ·i}{להעשות, להפוך ל־}
			\gl{por}{ל־}
			\gl{ŝton·o}{אבן}
			\gl{bitum·o}{„חֵמָר”, ביטומן, אספלט}
			\gl{kalk·o}{„חֹמֶר”}

		\item
			\gl{konstru·i}{לבנות}
			\gl{urb·o}{עיר}
			\gl{tur·o}{מגדל}
			\gl{ki·es}{של מי, ששלה/ו/ן/ם}
			\gl{supr·o}{פסגה, החלק העליון}
			\gl{ating·i}{להגיע}
			\gl{ĉiel·o}{שמיים}
			\gl{akir·i}{להשיג}
			\gl{glor·o}{תהילה}
			\gl{antaŭ}{לִפְנֵי}
			\gl{ol}{מ־, מאשר}
			\gl{dis·iĝ·i}{להתפזר}
			\gl{supr·aĵ·o}{פני־שטח}
			\gl{tut·a}{כול־ (הכל)}

		\item
			\gl{Etern·ul·o}{„הנצחי”}
			\gl{mal·lev·iĝ·i}{לרדת}
			\gl{vid·i}{לראות}
			\gl{ki·u}{ש־ (הוא)}
			\gl{hom·id·o}{בן־אדם}

		\item
			\gl{jen}{הִנֵּה}
			\gl{popol·o}{עַם}
			\gl{ĉi·u}{כול־ (כולם)}
			\gl{hav·i}{\L{have} באנגלית}
			\gl{ki·o}{מה (ש־)}
			\gl{komenc·i}{להתחיל}
			\gl{mal·help·i}{לטרפד}
			\gl{ĉi·o}{הכל, כל דבר}
			\gl{decid·i}{להחליט}

		\item
			\gl{do}{אם כן}
			\gl{konfuz·i}{לבלבל}
			\gl{ti·e}{שָׂם}
			\gl{ili·a}{שלהם}
			\gl{ke}{ש־}
			\gl{ne}{לא}
			\gl{kompren·i}{להבין}
			\gl{parol·o}{דיבור}

		\item
			\gl{dis·ig·i}{לפזר}
			\gl{ĉes·i}{להפסיק}

		\item
			\gl{ti·al}{בגלל זה}
			\gl{oni}{(נטול־גוף)}
			\gl{don·i}{לתת}
			\gl{ĝi}{זה, היא, הוא}
			\gl{nom·o}{שֵׁם}
			\gl{Babel}{בבל}
			\gl{ĉar}{כי}
	\end{enumerate}
\end{LTR}




\bisection{הומרניסמו}{{Homaranismo}}

כאן מופיעים רק שלושת הסעיפים הראשונים ב„הצהרה בדבר ההומרניסמו” (\EO{Deklaracio de Homaranismo}) מ־1913. את הטקסט כולו תוכלו לקרוא בוויקיטקסט: \url{eo.wikisource.org/wiki/Homaranismo}

ערך בוויקיפדיה העברית: \url{he.wikipedia.org/wiki/ומסינרמוה}

\begin{LTR}
	\begin{enumerate}[label=\protect\cxirkauxi{\arabic*}]
		\item\EO{Mi estas homo, kaj la tutan homaron mi rigardas kiel unu familion; la dividitecon de la homaro en diversajn reciproke malamikajn gentojn kaj gentreligiajn komunumojn mi rigardas kiel unu el la plej grandaj malfeliĉoj, kiu pli aŭ malpli frue devas malaperi kaj kies malaperon mi devas akceladi laŭ mia povo.}
		\item\EO{Mi vidas en ĉiu homo nur homon, kaj mi taksas ĉiun homon nur laŭ lia persona valoro kaj agoj.
			Ĉian ofendadon aŭ premadon de homo pro tio, ke li apartenas al alia gento, alia lingvo, alia religio aŭ alia socia klaso ol mi, mi rigardas kiel barbarecon.}
		\item\EO{Mi konscias, ke ĉiu lando apartenas ne al tiu aŭ alia gento, sed plene egalrajte al ĉiuj siaj loĝantoj, kian ajn supozatan devenon, lingvon, religion aŭ socian rolon ili havas; la identigadon de la interesoj de lando kun la interesoj de tiu aŭ alia gento aŭ religio kaj la pretekstadon de iaj historiaj rajtoj, kiuj permesas al unu gento en la lando regi super la aliaj gentoj kaj forrifuzi al ili la plej elementan kaj naturan rajton je la patrujo, mi rigardas kiel restaĵon el la tempoj barbaraj, kiam ekzistis nur rajto de pugno kaj glavo.}
	\end{enumerate}

	\begin{enumerate}[label=\protect\cxirkauxi{\arabic*}]
		\item
			\gl{mi}{אני}
			\gl{est·i}{להיות}
			\gl{hom·o}{אדם}
			\gl{kaj}{ו־}
			\gl{la}{ה־}
			\gl{tut·a}{כול־ (הכל)}
			\gl{hom·ar·o}{אנושות}
			\gl{rigard·i}{להביט, להחשיב}
			\gl{ki·el}{כ־; איך?}
			\gl{unu}{אחד}
			\gl{famili·o}{משפחה}
			\gl{divid·it·ec·o}{חלוקה, הפרדה}
			\gl{de}{של; מ־}
			\gl{en}{ב־; ל־ (עם \EO{-n})}
			\gl{divers·a}{נבדל}
			\gl{reciprok·e}{באופן הדדי}
			\gl{mal·amik·a}{עוין}
			\gl{gent·o}{עַם}
			\gl{gent·religi·a}{עם+דת+י}
			\gl{komun·um·o}{קהילה}
			\gl{el}{מ־}
			\gl{plej}{הכי}
			\gl{grand·a}{גדול}
			\gl{mal·feliĉ·o}{דבר מצער}
			\gl{ki·u}{ש־ (הוא)}
			\gl{pli}{יותר}
			\gl{aŭ}{או}
			\gl{mal·pli}{פחות}
			\gl{fru·e}{במוקדם}
			\gl{dev·i}{להיות חייב}
			\gl{mal·aper·i}{להעלם}
			\gl{ki·es}{של מי, ששלה/ו/ן/ם}
			\gl{mal·aper·o}{העלמות}
			\gl{akcel·ad·i}{להאיץ}
			\gl{laŭ}{לפי}
			\gl{mi·a}{שלי}
			\gl{pov·o}{יכולת}

		\item
			\gl{vid·i}{לראות}
			\gl{ĉi·u}{כול־ (כולם)}
			\gl{nur}{רק}
			\gl{taks·i}{להעריך}
			\gl{li·a}{שלו}
			\gl{person·a}{אישי}
			\gl{valor·o}{ערך}
			\gl{ag·o}{מעשה}
			\gl{ĉi·a}{כל סוג של־}
			\gl{ofend·ad·o}{פגיעה}
			\gl{prem·ad·o}{דיכוי}
			\gl{pro}{בעבור}
			\gl{ti·o}{זה}
			\gl{ke}{ש־}
			\gl{li}{הוא}
			\gl{aparten·i}{להשתייך}
			\gl{al}{אל, ל־}
			\gl{ali·a}{אַחֵר}
			\gl{lingv·o,}{שפה}
			\gl{religi·o}{דת}
			\gl{soci·a}{חברתי}
			\gl{klas·o}{מעמד}
			\gl{ol}{מ־, מאשר}
			\gl{barbar·ec·o}{ברבריות}

		\item
			\gl{konsci·i}{להכיר ב־, להיות ער ל־}
			\gl{land·o}{ארץ}
			\gl{ne}{לא}
			\gl{ti·u}{זה־}
			\gl{sed}{אלא, אך}
			\gl{plen·e}{באופן מלא}
			\gl{egal·rajt·e}{בשוויון־זכויות}
			\gl{si·a}{שלו/שלה/שלהם/ן עצמם}
			\gl{loĝ·i}{לגור}
			\gl{ki·a ajn}{איזשהו (לא משנה)}
			\gl{supoz·at·a}{משוער}
			\gl{de·ven·o}{מוצא}
			\gl{rol·o}{תפקיד, מקום}
			\gl{ili}{הם}
			\gl{hav·i}{\L{have} באנגלית}
			\gl{ident·ig·ad·o}{זיהוי}
			\gl{interes·o}{אינטרס}
			\gl{kun}{עִם}
			\gl{pretekst·ad·o}{תירוץ}
			\gl{i·a}{איזשהו סוג של־}
			\gl{histori·a}{היסטורי}
			\gl{rajt·o}{זכות}
			\gl{permes·i}{להתיר}
			\gl{reg·i}{למשול}
			\gl{super}{מעל}
			\gl{for·rifuz·i}{לשלול}
			\gl{element·a}{בסיסי}
			\gl{natur·a}{טבעי}
			\gl{je}{(מילת־יחס כללית)}
			\gl{patr·uj·o}{מולדת, ארץ־האבות}
			\gl{rest·aĵ·o}{שריד, שארית}
			\gl{temp·o}{זמן}
			\gl{barbar·a}{ברברי}
			\gl{ki·am}{כש־, מתי}
			\gl{ekzist·i}{להתקיים}
			\gl{pugn·o}{אגרוף}
			\gl{glav·o}{חרב}
	\end{enumerate}
\end{LTR}



\bisection{התקווה}{{La Espero}}

\begin{multicols}{2}
	מילים: אליעזר לודוויג זמנהוף\\
	לחן: פליסיאן מני דה מניל

	ביצוע, עם כתוביות: \url{youtu.be/mk-0D3D75Ao}\\
	בוויקיפדיה: \url{he.wikipedia.org/wiki/La_Espero}
\end{multicols}

\begin{LTR}
	\begin{multicols}{2}
		\begin{enumerate}
				\versperanto{1}{En la mondon venis nova sento,\\
					tra la mondo iras forta voko;\\
					per flugiloj de facila vento\\
				nun de loko flugu ĝi al loko.}

				\versperanto{2}{Ne al glavo sangon soifanta\\
					ĝi la homan tiras familion:\\
					al la mond' eterne militanta\\
				ĝi promesas sanktan harmonion.}

				\versperanto{3}{Sub la sankta signo de l' espero\\
					kolektiĝas pacaj batalantoj,\\
					kaj rapide kreskas la afero\\
				per laboro de la esperantoj.}

				\versperanto{4}{Forte staras muroj de miljaroj\\
					inter la popoloj dividitaj;\\
					sed dissaltos la obstinaj baroj,\\
				per la sankta amo disbatitaj.}

				\versperanto{5}{Sur neŭtrala lingva fundamento,\\
					komprenante unu la alian,\\
					la popoloj faros en konsento\\
				unu grandan rondon familian.}

				\versperanto{6}{Nia diligenta kolegaro\\
					en laboro paca ne laciĝos,\\
					ĝis la bela sonĝo de l' homaro\\
				por eterna ben' efektiviĝos.}
		\end{enumerate}
	\end{multicols}

	\begin{enumerate}[label=\protect\cxirkauxi{\arabic*}]
		\item
			\gl{en}{ב־; ל־ (עם \EO{-n})}
			\gl{la}{ה־}
			\gl{mond·o}{עולם}
			\gl{ven·i}{לבוא}
			\gl{nov·a}{חדש}
			\gl{sent·o}{תחושה}
			\gl{tra}{דרך}
			\gl{ir·i}{ללכת}
			\gl{fort·a}{חזק}
			\gl{voko}{קול}
			\gl{per}{על־ידי}
			\gl{flug·il·o}{כנף}
			\gl{de}{של; מ־}
			\gl{facil·a}{קל}
			\gl{vent·o}{רוח}
			\gl{nun}{עכשיו}
			\gl{lok·o}{מקום}
			\gl{flug·i}{לעוף}
			\gl{ĝi}{זה, היא, הוא}
			\gl{al}{אל, ל־}

		\item
			\gl{ne}{לא}
			\gl{glav·o}{חרב}
			\gl{sang·o}{דם}
			\gl{soif·i}{להיות צמא}
			\gl{hom·a}{אנושי}
			\gl{tir·i}{למשוך}
			\gl{famili·o}{משפחה}
			\gl{etern·e}{תמידית}
			\gl{milit·i}{להלחם}
			\gl{promes·i}{להבטיח}
			\gl{sankt·a}{קודש}
			\gl{harmoni·o}{הרמוניה}

		\item
			\gl{sub}{תחת}
			\gl{sign·o}{סימן}
			\gl{esper·o}{תקווה}
			\gl{kolekt·iĝ·i}{להתאסף}
			\gl{pac·a}{של שלום}
			\gl{batal·i}{להלחם, להאבק}
			\gl{kaj}{ו־}
			\gl{rapid·e}{מהר}
			\gl{kresk·i}{לגדול}
			\gl{afer·o}{עניין}
			\gl{labor·o}{עבודה}
			\gl{esper·i}{לקוות}

		\item
			\gl{fort·e}{בחוזקה}
			\gl{star·i}{לעמוד}
			\gl{mur·o}{חומה}
			\gl{mil·jar·o}{אלף שנים}
			\gl{inter}{בין}
			\gl{popol·o}{עַם}
			\gl{divid·it·a}{מחולק, מופרד}
			\gl{sed}{אלא, אך}
			\gl{dis·salt·i}{להפריד+לקפוץ}
			\gl{obstin·a}{עיקש}
			\gl{bar·o}{מחסום}
			\gl{amo}{אהבה}
			\gl{dis·bat·it·a}{מרוסק}

		\item
			\gl{sur}{על}
			\gl{neŭtr·al·a}{נייטרלי}
			\gl{lingv·a}{לשוני}
			\gl{fundament·o}{בסיס}
			\gl{kompren·i}{להבין}
			\gl{unu}{אחד}
			\gl{ali·a}{אַחֵר}
			\gl{far·i}{לעשות}
			\gl{konsent·o}{הסכמה}
			\gl{grand·a}{גדול}
			\gl{rond·o}{מעגל}
			\gl{famili·a}{משפחתי}

		\item
			\gl{ni·a}{שלנו}
			\gl{diligent·a}{חרוץ}
			\gl{koleg·ar·o}{צוות}
			\gl{lac·iĝ·i}{להתעייף}
			\gl{ĝis}{עד (ש־)}
			\gl{bel·a}{יפה}
			\gl{sonĝ·o}{חלום}
			\gl{hom·ar·o}{אנושות}
			\gl{por}{ל־}
			\gl{etern·a}{נצחי}
			\gl{ben·o}{בְּרָכָה}
			\gl{efektiv·iĝ·i}{להתגשם}
	\end{enumerate}
\end{LTR}



\bisection{קופר האספרנטיסט הקטן}{{Kupero la esperantisteto}}

לצפיה: \url{youtu.be/l0ErKbLL5WQ}

\begin{LTR}
	\EO{Kie estas via buŝo? Kie estas viaj oreloj? Kie estas viaj haroj? Kie estas viaj okuloj? Kie estas via nazo? Kie estas via lango? Kie estas viaj manoj? Kie estas via umbiliko? Kie estas viaj piedoj? Kie estas via ĉemizo? Kion diras kato? —Baaaŭ. Kion diras hundo? —Uŭ uŭ uŭ. Kion diras bovo? —Muuu. Ĉu vi povas kisi min? Kisi, kisi, kisi…}

	\gl{ki·e}{איפה}
	\gl{est·i}{להיות}
	\gl{vi·a}{שלך/ן/ם}
	\gl{buŝ·o}{פֶּה}
	\gl{orel·o}{אוזן}
	\gl{har·o}{שערה}
	\gl{okul·o}{עין}
	\gl{naz·o}{אף}
	\gl{lango}{לשון}
	\gl{man·o}{יד}
	\gl{umbilik·o}{פופיק}
	\gl{pied·o}{כף־רגל}
	\gl{ĉemiz·o}{חולצה}
	\gl{ki·o}{מה}
	\gl{dir·i}{לומר, להגיד}
	\gl{kat·o}{חתול}
	\gl{hund·o}{כלב}
	\gl{bov·o}{פר/ה}
	\gl{ĉu}{האם}
	\gl{vi}{את/ה/ן/ם}
	\gl{pov·i}{להיות יכול/ה}
	\gl{kis·i}{לנשק}
	\gl{mi}{אני}
\end{LTR}



\bisection{כך מסתובב העולם}{{Tiel la Mondo Iras}}

מאת \L{Juliano Hernández Angulo}. לצפיה בביצוע שלו לשיר: \url{vimeo.com/19634742}

\begin{LTR}
	\begin{multicols}{2}
		\begin{enumerate}
				\versperanto{1}{Tiel la mondo iras …}

				\versperanto{2}{Lundo, merkredo, sabato,\\
					mardo, ĵaŭdo kaj dimanĉo.\\
					Jen milito, jen la paco,\\
					jen infano kun malsato.\\
					La misiloj preskaŭ falas,\\
					la kolomboj malkonsentas,\\
				Estas tiel, estas tiel.}

				\versperanto{3}{Jen virino kiu ne sidas,\\
					ĉar laboro ĉiam estas,\\
					kaj la patro kiu ne alvenas,\\
					ĉar la poŝo estas malplena.\\
					Tiom da manoj kiuj konstruas,\\
					kaj la aliaj kiuj detruas,\\
				Estas tiel, estas tiel.}

			\begin{verse}{Tiel la mondo iras …}\end{verse}

				\versperanto{4}{Dekses horoj kiuj sonoras\\
					kaj ok horoj kiuj silentas\\
					Multaj homoj kiuj rapidas,\\
					jam la alia tago venas.\\
					Ni profitu la momenton,\\
					ĉar la vivo ne atendas,\\
				Estas tiel, estas tiel.}

			\begin{verse}{Lundo, merkredo, sabato, …}\end{verse}

			\begin{verse}{Tiel la mondo iras …}\end{verse}

				\versperanto{5}{Iras por mi, iras por vi,\\
					iras por ŝi, iras por li, tiel la mondo,\\
					iras por mi, iras por vi,\\
				iras por ŝi, iras por li, tiel la mondo.}

			\begin{verse}{Tiel la mondo iras …}\end{verse}
		\end{enumerate}
	\end{multicols}

	\begin{enumerate}[label=\protect\cxirkauxi{\arabic*}]
		\item
			\gl{ti·el}{כך}
			\gl{la}{ה־}
			\gl{mond·o}{עולם}
			\gl{ir·i}{ללכת}

		\item
			\gl{lund·o}{יום שני}
			\gl{merkred·o}{יום רביעי}
			\gl{sabat·o}{שבת}
			\gl{mard·o}{יום שלישי}
			\gl{ĵaŭd·o}{יום חמישי}
			\gl{kaj}{ו־}
			\gl{dimanĉ·o}{יום ראשון}
			\gl{jen}{הִנֵּה}
			\gl{milit·o}{מלחמה}
			\gl{pac·o}{שלום}
			\gl{infan·o}{ילד}
			\gl{kun}{עִם}
			\gl{mal·sat·o}{רעב}
			\gl{misil·o}{טיל}
			\gl{preskaŭ}{כמעט}
			\gl{fal·i}{ליפול}
			\gl{kolomb·o}{יונה}
			\gl{mal·konsent·i}{להתנגד}
			\gl{est·i}{להיות}

		\item
			\gl{vir·in·o}{אשה}
			\gl{ki·u}{ש־ (הוא)}
			\gl{ne}{לא}
			\gl{sid·i}{לשבת}
			\gl{ĉar}{כי}
			\gl{labor·o}{עבודה}
			\gl{ĉi·am}{תמיד}
			\gl{patr·o}{אב, הוֹרֶה}
			\gl{al·ven·i}{להגיע}
			\gl{poŝ·o}{כיס}
			\gl{mal·plen·a}{ריק}
			\gl{ti·om}{כזאת כמות}
			\gl{da}{של (מחבר כמות)}
			\gl{man·o}{יד}
			\gl{konstrua·i}{לבנות}
			\gl{ali·a}{אַחֵר}
			\gl{detrua·i}{להרוס}

		\item
			\gl{dek·ses}{16}
			\gl{hor·o}{שעה}
			\gl{sonor·i}{להשמיע צליל}
			\gl{ok}{8}
			\gl{silent·i}{לשתוק, להיות שקט}
			\gl{mult·a}{הרבה, רב}
			\gl{hom·o}{אדם}
			\gl{rapid·i}{למהר}
			\gl{jam}{כבר}
			\gl{tag·o}{יום}
			\gl{ven·i}{לבוא}
			\gl{ni}{אנחנו}
			\gl{profit·i}{להרוויח}
			\gl{moment·o}{רֶגַע}
			\gl{viv·o}{חיים}
			\gl{atend·i}{לחכות}

		\item
			\gl{por}{ל־}
			\gl{mi}{אני}
			\gl{vi}{את/ה/ן/ם}
			\gl{ŝi}{היא}
			\gl{li}{הוא}
	\end{enumerate}
\end{LTR}



\bisection{פסקאות פותחות מספרים מוכרים}{Komencaj alineoj de famaj libroj}

\newcommand{\sourcetarget}[6]{מקור: \L{#1, #2, #3}.\\תרגום: \L{#4, #5, #6}.}



\bisubsection{התנ״ך}{La Biblio}

*** %http://www.steloj.de/esperanto/biblio/

\begin{tabulary}{\textwidth}{CRLC}
	\verso{1}{בְּרֵאשִׁ֖ית בָּרָ֣א אֱלֹהִ֑ים אֵ֥ת הַשָּׁמַ֖יִם וְאֵ֥ת הָאָֽרֶץ׃}{En la komenco Dio kreis la ĉielon kaj la teron.}
	\verso{2}{וְהָאָ֗רֶץ הָיְתָ֥ה תֹ֙הוּ֙ וָבֹ֔הוּ וְחֹ֖שֶׁךְ עַל־פְּנֵ֣י תְה֑וֹם וְר֣וּחַ אֱלֹהִ֔ים מְרַחֶ֖פֶת עַל־פְּנֵ֥י הַמָּֽיִם׃}{Kaj la tero estis senforma kaj dezerta, kaj mallumo estis super la abismo; kaj la spirito de Dio ŝvebis super la akvo.}
	\verso{3}{וַיֹּ֥אמֶר אֱלֹהִ֖ים יְהִ֣י א֑וֹר וַֽיְהִי־אֽוֹר׃}{Kaj Dio diris: Estu lumo; kaj fariĝis lumo.}
	\verso{4}{וַיַּ֧רְא אֱלֹהִ֛ים אֶת־הָא֖וֹר כִּי־ט֑וֹב וַיַּבְדֵּ֣ל אֱלֹהִ֔ים בֵּ֥ין הָא֖וֹר וּבֵ֥ין הַחֹֽשֶׁךְ׃}{Kaj Dio vidis la lumon, ke ĝi estas bona; kaj Dio apartigis la lumon de la mallumo.}
	\verso{5}{וַיִּקְרָ֨א אֱלֹהִ֤ים ׀ לָאוֹר֙ י֔וֹם וְלַחֹ֖שֶׁךְ קָ֣רָא לָ֑יְלָה וַֽיְהִי־עֶ֥רֶב וַֽיְהִי־בֹ֖קֶר י֥וֹם אֶחָֽד׃}{Kaj Dio nomis la lumon Tago, kaj la mallumon Li nomis Nokto. Kaj estis vespero, kaj estis mateno, unu tago.}
\end{tabulary}

\begin{LTR}
	\begin{enumerate}[label=\protect\cxirkauxi{\arabic*}]
		\item
			\gl{en}{ב־, ל־ (עם \EO{-n})}
			\gl{la}{ה־}
			\gl{komenc·o}{התחלה}
			\gl{di·o}{אלוהים}
			\gl{kre·i}{ליצור}
			\gl{ĉiel·o}{שמיים}
			\gl{kaj}{ו־}
			\gl{ter·o}{ארץ, אדמה}
		\item
			\gl{est·i}{להיות}
			\gl{sen·form·a}{חסר־צורה}
			\gl{dezert·a}{שומם}
			\gl{mal·lum·o}{חושך}
			\gl{super}{מעל}
			\gl{abism·o}{תהום}
			\gl{spirit·o}{רוח, נשמה}
			\gl{de}{של; מ־}
			\gl{ŝveb·i}{לרחף}
			\gl{akv·o}{מים}
		\item
			\gl{dir·i}{לומר, להגיד}
			\gl{far·iĝ·i}{להעשות, להפוך ל־}
			\gl{lum·o}{אור}
		\item
			\gl{vid·i}{לראות}
			\gl{ke}{ש־}
			\gl{ĝi}{זה, היא, הוא}
			\gl{bon·a}{טוב}
			\gl{apart·ig·i}{להפריד}
		\item
			\gl{nom·i}{לקרוא בשם}
			\gl{tag·o}{יום}
			\gl{li}{הוא}
			\gl{nokt·o}{לילה}
		\item
			\gl{vesper·o}{עֶרֶב}
			\gl{maten·o}{בֹּקֶר}
			\gl{unu}{אחד}
	\end{enumerate}
\end{LTR}


\bisubsection{גאווה ודעה קדומה}{Fiereco kaj Anta\ŭ ju\ĝ emo}

\sourcetarget{\textsc{Austen}, Jane}{Pride and Prejudice}{1813}
{\textsc{Harlow}, Donald J.}{Fiereco kaj Antaŭjuĝemo}{***}

\begin{LTR}
	\begin{tabulary}{\textwidth}{LL}
		\biling{Estas veraĵo universale agnoskata, ke fraŭlo kun bonhavo certe bezonas edzinon.}
		{It is a truth universally acknowledged, that a single man in possession of a good fortune, must be in want of a wife.}
		\biling{Kiom ajn malmulte oni konas la sentojn aŭ opiniojn de tia homo kiam unuan fojon li alvenas en la distrikto, tiu veraĵo estas tiel bone fiksita en la mensoj de la apudaj familioj, ke ili ja supozas lin la rajtigita posedaĵo de iu aŭ alia elinter iliaj filinoj.}
		{However little known the feelings or views of such a man may be on his first entering a neighbourhood, this truth is so well fixed in the minds of the surrounding families, that he is considered the rightful property of some one or other of their daughters.}
	\end{tabulary}

	\gl{est·i}{להיות}
	\gl{ver·aĵ·o}{עובדה, אמיתה}
	\gl{universal·e}{באופן אוניברסלי***}
	\gl{agnosk·i}{להכיר (במשהו)}
	\gl{ke}{ש־}
	\gl{fraŭl·o}{רווק}
	\gl{kun}{עִם}
	\gl{bonhavo}{***}
	\gl{certe}{לבטח}
	\gl{bezon·i}{להיות צריך}
	\gl{edz·in·o}{רעיה}

	\gl{Kiom}{}
	\gl{ajn}{}
	\gl{malmulte}{}
	\gl{oni}{}
	\gl{konas}{}
	\gl{la}{}
	\gl{sentojn}{}
	\gl{aŭ}{}
	\gl{opiniojn}{}
	\gl{de}{}
	\gl{tia}{}
	\gl{homo}{}
	\gl{kiam}{}
	\gl{unuan}{}
	\gl{fojon}{}
	\gl{li}{}
	\gl{alvenas}{}
	\gl{en}{}
	\gl{la}{}
	\gl{distrikto,}{}
	\gl{tiu}{}
	\gl{veraĵo}{}
	\gl{estas}{}
	\gl{tiel}{}
	\gl{bone}{}
	\gl{fiksita}{}
	\gl{en}{}
	\gl{la}{}
	\gl{mensoj}{}
	\gl{de}{}
	\gl{la}{}
	\gl{apudaj}{}
	\gl{familioj,}{}
	\gl{ke}{}
	\gl{ili}{}
	\gl{ja}{}
	\gl{supozas}{}
	\gl{lin}{}
	\gl{la}{}
	\gl{rajtigita}{}
	\gl{posedaĵo}{}
	\gl{de}{}
	\gl{iu}{}
	\gl{aŭ}{}
	\gl{alia}{}
	\gl{elinter}{}
	\gl{iliaj}{}
	\gl{filinoj.}{}
\end{LTR}



\bisubsection{אליס בארץ הפלאות}{Alico en Mirlando}

\sourcetarget{\textsc{Carrol}, Lewis}{Alice's Adventures in Wonderland}{1865}
{\textsc{Broadribb}, Donald}{Alico en Mirlando}{1996}

\begin{LTR}
	\begin{tabulary}{\textwidth}{LL}
		\biling{Alicon komencis multe tedi la sidado apud sia fratino sur la bordo de la rivereto, kaj la manko de io farinda; unu-du-foje ŝi rigardetis transŝultre la libron kiun legas ŝia fratino, sed ĝi havis nek bildojn nek konversaciojn, «kaj kiel utilas libro,» pensis Alico, «sen bildoj aŭ konversacioj?»}
		{Alice was beginning to get very tired of sitting by her sister on the bank, and of having nothing to do: once or twice she had peeped into the book her sister was reading, but it had no pictures or conversations in it, ‘and what is the use of a book,’ thought Alice ‘without pictures or conversation?’}
	\end{tabulary}

	\gl{komenc·i}{להתחיל}
	\gl{mult·e}{במידה רבה}
	\gl{ted·i}{לשעמם}
	\gl{la}{ה־}
	\gl{sid·ad·o}{ישיבה}
	\gl{apud}{לצד}
	\gl{si·a}{שלו/שלה/שלהם/ן עצמם}
	\gl{frat·in·o}{אחות}
	\gl{sur}{על}
	\gl{bord·o}{קצה, גדה, חוף}
	\gl{de}{של; מ־}
	\gl{river·et·o}{נחל}
	\gl{kaj}{ו־}
	\gl{mank·o}{העדר}
	\gl{i·o}{משהו}
	\gl{far·ind·a}{שראוי לעשות}
	\gl{unu-du-foj·e}{פעם־פעמיים}
	\gl{ŝi}{היא}
	\gl{rigard·et·i}{להציץ}
	\gl{trans·ŝultr·e}{מעבר לכתף}
	\gl{libr·o}{סֵפֶר}
	\gl{ki·u}{ש־ (הוא)}
	\gl{leg·i}{לקרוא}
	\gl{ŝi·a}{שלה}
	\gl{sed}{אֲבָל}
	\gl{ĝi}{זה, היא, הוא}
	\gl{hav·i}{\L{have} באנגלית}
	\gl{nek~… nek}{לא … ולא}
	\gl{bild·o}{תמונה}
	\gl{konversaci·o,}{שיחה}
	\gl{ki·el}{כ־; איך?}
	\gl{util·i}{להועיל}
	\gl{pens·i}{לחשוב}
	\gl{sen}{בלי}
\end{LTR}



\bisubsection{ההוביט}{La Hobito}

\sourcetarget{\textsc{Tolkien}, J. R. R.}{The Hobbit}{1937}
{***}{La Hobito}{***}

\begin{LTR}
	\begin{tabulary}{\textwidth}{LL}
		\biling{En truo en tero vivis hobito. Ne aĉa, malpura, malseka truo plena je vermstumpoj kaj ŝlima odoro, nek seka, dezerta, sabla truo sen sidlokoj aŭ manĝaĵoj. Ĝi estis hobito-truo, kaj tio signifas komforton.}
		{In a hole in the ground there lived a hobbit. Not a nasty, dirty, wet hole, filled with the ends of worms and an oozy smell, nor yet a dry, bare, sandy hole with nothing in it to sit down on or to eat: it was a hobbit-hole, and that means comfort.}
	\end{tabulary}

	\gl{en}{ב־; ל־ (עם \EO{-n})}
	\gl{tru·o}{חור}
	\gl{ter·o}{ארץ, אדמה}
	\gl{viv·i}{לִחיות}
	\gl{hobit·o}{הוביט}
	\gl{ne}{לא}
	\gl{aĉ·a}{מגעיל}
	\gl{mal·pur·a}{מלולכך}
	\gl{mal·sek·a}{רטוב}
	\gl{plen·a}{מלא}
	\gl{je}{(מילת־יחס כללית)}
	\gl{vermstumpo}{***}
	\gl{kaj}{ו־}
	\gl{ŝlim·a}{בוצי}
	\gl{odor·o}{ריח}
	\gl{nek}{וגם לא}
	\gl{sek·a}{יבש}
	\gl{dezert·a}{שומם}
	\gl{sabl·a}{חולי}
	\gl{sen}{בלי}
	\gl{sid·lok·o}{מקום לשבת}
	\gl{aŭ}{או}
	\gl{manĝ·aĵ·o}{מאכל}
	\gl{ĝi}{זה, היא, הוא}
	\gl{est·i}{להיות}
	\gl{hobit·o-tru·o}{מאורת־הוביט}
	\gl{ti·o}{זה}
	\gl{signif·i}{להיות עם משמעות (של)}
	\gl{komfort·o}{נוחות}
\end{LTR}



\bisubsection{אלף תשע־מאות שמונים וארבע}{Mil Naŭcent Okdek Kvar}

\sourcetarget
{\textsc{Orwell}, George}{Nineteen Eighty-Four}{1949}
{\textsc{Broadribb}, Donald}{Mil Naŭcent Okdek Kvar}{2001}

\begin{LTR}
	\begin{tabulary}{\textwidth}{LL}
		\biling{Estis hela malvarma tago en aprilo, kaj la horloĝoj sonigis la dektrian horon. Winston Smith, kun la mentono premita en la bruston, por eskapi de la akrega vento, rapide puŝis sin tra la vitrajn pordojn de la Loĝejoj de la Venko, kvankam ne sufiĉe rapide por neebligi la eniron kun li de nebuleto de eroplena polvo.}
		{It was a bright cold day in April, and the clocks were striking thirteen. Winston Smith, his chin nuzzled into his breast in an effort to escape the vile wind, slipped quickly through the glass doors of Victory Mansions, though not quickly enough to prevent a swirl of gritty dust from entering along with him. }
	\end{tabulary}

	\gl{est·i}{להיות}
	\gl{hel·a}{בהיר}
	\gl{mal·varm·a}{קר}
	\gl{tag·o}{יום}
	\gl{en}{ב־; ל־ (עם \EO{-n})}
	\gl{april·o}{אפריל}
	\gl{kaj}{ו־}
	\gl{la}{ה־}
	\gl{horloĝ·o}{שעון}
	\gl{son·ig·i}{***}
	\gl{dek·tri·a}{ה־13}
	\gl{hor·o}{שעה}
	\gl{kun}{עִם}
	\gl{mentono}{}
	\gl{premita}{}
	\gl{en}{}
	\gl{la}{}
	\gl{bruston,}{}
	\gl{por}{}
	\gl{eskapi}{}
	\gl{de}{}
	\gl{la}{}
	\gl{akrega}{}
	\gl{vento,}{}
	\gl{rapide}{}
	\gl{puŝis}{}
	\gl{sin}{}
	\gl{tra}{}
	\gl{la}{}
	\gl{vitrajn}{}
	\gl{pordojn}{}
	\gl{de}{}
	\gl{la}{}
	\gl{Loĝejoj}{}
	\gl{de}{}
	\gl{la}{}
	\gl{Venko,}{}
	\gl{kvankam}{}
	\gl{ne}{}
	\gl{sufiĉe}{}
	\gl{rapide}{}
	\gl{por}{}
	\gl{neebligi}{}
	\gl{la}{}
	\gl{eniron}{}
	\gl{kun}{}
	\gl{li}{}
	\gl{de}{}
	\gl{nebuleto}{}
	\gl{de}{}
	\gl{eroplena}{}
	\gl{polvo.}{}
\end{LTR}



%\bisubsection{הארי פוטר ואבן החכמים}{Harry Potter kaj la Ŝtono de la Saĝuloj}
\bisubsection{הארי פוטר ואבן החכמים}{Harry Potter kaj la ŝtono de la Saĝuloj}

\sourcetarget{\textsc{Rowling}, J. K.}{Harry Potter and the Philosopher's Stone}{1997}
{***}{Harry Potter kaj la ŝtono de la Saĝuloj}{***}

\begin{LTR}
	\begin{tabulary}{\textwidth}{LL}
		\biling{Gesinjoroj Dursli ĉe numero kvar, Ligustra Vojo, fieris diri, ke ili estas «perfekte normalaj, multan dankon.» Neniu povus imagi ilin implikiĝi en io stranga aŭ mistera. Ne estis loko por tiaj sensencaĵoj en ilia vivo.}
		{Mr and Mrs Dursley, of number four, Privet Drive, were proud to say that they were perfectly normal, thank you very much. They were the last people you’d expect to be involved in anything strange or mysterious, because they just didn’t hold with such nonsense.}
	\end{tabulary}
	%
	%\biling{Sinjoro Dursli estis direktoro ĉe la firmao Grunings, kiu fabrikis borilojn. Li estis granda, boveca viro preskaŭ sen kolo, sed kun tre larĝa liphararo. Sinjorino Dursli estis maldika kaj blonda, kaj havis kolon duoble pli longan ol la normalan, kio tre utilis ĉar ŝi pasigis tiom da tempo etendante ĝin super la ĝardenbarilojn, spionante la najbarojn. Gesinjoroj Dursli havis fileton nomitan Dadli, kaj laŭ ili ne ekzistis, ie ajn, pli aminda knabo.}


	\gl{ge·sinjor·o}{אדון/גברת}
	\gl{ĉe}{}
	\gl{numer·o}{מִסְפָּר}
	\gl{kvar}{4}
	\gl{Ligustra}{***}
	\gl{Vojo}{דֶּרֶךְ}
	\gl{fier·i}{להיות גאה}
	\gl{dir·i}{לומר, להגיד}
	\gl{ke}{ש־}
	\gl{ili}{הם}
	\gl{est·i}{להיות}
	\gl{perfekt·e}{לגמרי, באופן מושלם}
	\gl{norm·al·a}{נורמלי}
	\gl{mult·a}{הרבה, רב}
	\gl{dank·o}{תודה}
	\gl{neni·u}{אף אחד}
	\gl{pov·i}{להיות יכול/ה}
	\gl{imag·i}{לדמיין}
	\gl{implik·iĝ·i}{להיות מעורב}
	\gl{en}{ב־; ל־ (עם \EO{-n})}
	\gl{i·o}{משהו}
	\gl{strang·a}{מוזר}
	\gl{aŭ}{או}
	\gl{mister·a}{מיסטורי}
	\gl{ne}{לא}
	\gl{lok·o}{מקום}
	\gl{por}{ל־}
	\gl{ti·a}{כזה}
	\gl{sen·senc·aĵ·o}{שטות}
	\gl{ili·a}{שלהם}
	\gl{viv·o}{חיים}

	% \gl{Sinjoro}{}
	% \gl{Dursli}{}
	% \gl{estis}{}
	% \gl{direktoro}{}
	% \gl{ĉe}{}
	% \gl{la}{}
	% \gl{firmao}{}
	% \gl{Grunings,}{}
	% \gl{kiu}{}
	% \gl{fabrikis}{}
	% \gl{borilojn.}{}
	% \gl{Li}{}
	% \gl{estis}{}
	% \gl{granda,}{}
	% \gl{boveca}{}
	% \gl{viro}{}
	% \gl{preskaŭ}{}
	% \gl{sen}{}
	% \gl{kolo,}{}
	% \gl{sed}{}
	% \gl{kun}{}
	% \gl{tre}{}
	% \gl{larĝa}{}
	% \gl{liphararo.}{}
	% \gl{Sinjorino}{}
	% \gl{Dursli}{}
	% \gl{estis}{}
	% \gl{maldika}{}
	% \gl{kaj}{}
	% \gl{blonda,}{}
	% \gl{kaj}{}
	% \gl{havis}{}
	% \gl{kolon}{}
	% \gl{duoble}{}
	% \gl{pli}{}
	% \gl{longan}{}
	% \gl{ol}{}
	% \gl{la}{}
	% \gl{normalan,}{}
	% \gl{kio}{}
	% \gl{tre}{}
	% \gl{utilis}{}
	% \gl{ĉar}{}
	% \gl{ŝi}{}
	% \gl{pasigis}{}
	% \gl{tiom}{}
	% \gl{da}{}
	% \gl{tempo}{}
	% \gl{etendante}{}
	% \gl{ĝin}{}
	% \gl{super}{}
	% \gl{la}{}
	% \gl{ĝardenbarilojn,}{}
	% \gl{spionante}{}
	% \gl{la}{}
	% \gl{najbarojn.}{}
	% \gl{Gesinjoroj}{}
	% \gl{Dursli}{}
	% \gl{havis}{}
	% \gl{fileton}{}
	% \gl{nomitan}{}
	% \gl{Dadli}{}
	% \gl{kaj}{}
	% \gl{laŭ}{}
	% \gl{ili}{}
	% \gl{ne}{}
	% \gl{ekzistis,}{}
	% \gl{ie}{}
	% \gl{ajn,}{}
	% \gl{pli}{}
	% \gl{aminda}{}
	% \gl{knabo.}{}
\end{LTR}
