\bichapter{דקדוק}{Gramatiko}
\bisection{אלף־בית}{{Alfabeto}}

קריאה של האלף־בית: \url{omniglot.com/writing/esperanto.htm}

\begin{center}
	\setLR
	\gloseto{a}{אַ}
	\gloseto{b}{בּ}
	\gloseto{c}{צ}
	\gloseto{ĉ}{צ׳}
	\gloseto{d}{ד}
	\gloseto{e}{אֶ}
	\gloseto{f}{פ}
	\gloseto{g}{ג}
	\gloseto{ĝ}{ג׳}
	\gloseto{h}{ה}
	\gloseto{ĥ}{כֿ}
	\gloseto{i}{אִ}
	\gloseto{j}{י}
	\gloseto{ĵ}{ז׳}\\
	\gloseto{k}{ק}
	\gloseto{l}{ל}
	\gloseto{m}{מ}
	\gloseto{n}{נ}
	\gloseto{o}{אֹ}
	\gloseto{p}{פֿ}
	\gloseto{r}{ר}
	\gloseto{s}{ס}
	\gloseto{ŝ}{ש}
	\gloseto{t}{ט}
	\gloseto{u}{אֻ}
	\gloseto{ŭ}{וו}
	\gloseto{v}{בֿ}
	\gloseto{z}{ז}
\end{center}



\bisection{דקדוק בסיסי}{{Baza gramatiko}}

\begin{LTR}
	\begin{enumerate}[label=\protect\cxirkauxi{\arabic*}, nosep]
		\item \gl{libr·o}{סֵפֶר} \gl{la libr·o}{הספר} \gl{libr·o·j}{ספרים} \gl{la libr·o·j}{הספרים}

			\begin{gramatiko}
			\gramitem שמות־עצם מסתיימים ב־\EO{-o}
			\gramitem תווית מיידעת: \EO{la}; תווית מסתמת: {\fontspec{Symbola}∅} (אֶפֶס)
			\gramitem ריבוי: \EO{-j}
			\end{gramatiko}

		\item\gl{libr·o·n}{ספר (כמושא)} \gl{la libr·o·n}{את הספר} \gl{libr·o·j·n}{ספרים (כמושאים)} \gl{la libr·o·j·n}{את הספרים}

			\begin{gramatiko}
			\gramitem מושא: \EO{-n}
			\end{gramatiko}

		\item\raggedright \gl{ruĝ·a libr·o}{ספר אדום} \gl{la ruĝ·a·j libr·o·j}{הספרים האדומים} \gl{la ruĝ·a·j·n libr·o·j·n}{את הספרים האדומים}

			\begin{gramatiko}
				\gramitem שמות־תואר מסתיימים ב־\EO{-a}
				\gramitem התאם במספר וביחסה
			\end{gramatiko}

		\item\raggedright \gl{mi}{אני} \gl{ni}{אנחנו} \gl{(ci)}{את/ה} \gl{vi}{את/ה, אתן/ם} \gl{ĝi}{זה, היא, הוא} \gl{ŝi}{היא} \gl{li}{הוא} \gl{ili}{הן/ם} \gl{oni}{(נטול־גוף)} \gl{si}{עצמה/ו/ן/ם}

			\begin{gramatiko}
				\gramitem הכינוי \EO{ci}, שמציין גוף שני יחיד ***, כמעט ולא נמצא בשימוש; בפועל משתמשים ב־\EO{vi} הן ליחיד והן לרבים (בדומה ל־\L{you} באנגלית)
			\end{gramatiko}

		\item\raggedright \gl{mi·n}{אותי} \gl{ili·n}{אותן/ם}

			\begin{gramatiko}
			\gramitem מושא של כינויי־גוף: כמו של שמות־עצם
			\end{gramatiko}

		\item\raggedright \gl{mi·a libr·o}{הספר שלי} \gl{mi·a·j libr·o·j}{הספרים שלי} \gl{mi·a·n libr·o·n}{את הספר שלי} \gl{mi·a·j·n libr·o·j·n}{את הספרים שלי}

			\begin{gramatiko}
			\gramitem שייכות של כינויי־גוף: תוספת \EO{-a} עם התאם (כמו תארים)
			\end{gramatiko}

		\item\raggedright \gl{unu}{1} \gl{du}{2} \gl{tri}{3} \gl{kvar}{4} \gl{kvin}{5} \gl{ses}{6} \gl{sep}{7} \gl{ok}{8} \gl{naŭ}{9} \gl{dek}{10} \gl{cent}{100} \gl{mil}{1000} \gl{miliono}{1000000} \gl{miliardo}{1000000000} \gl{kvin·cent tri·dek tri}{533} \gl{la unu·a libr·o}{הספר הראשון} \gl{la du·a libr·o}{הספר השני}

			\begin{gramatiko}
			\gramitem מערכת ספירה עשרונית ***
			\gramitem מספרים סודרים: תוספת \EO{-a} (תואר)
			\end{gramatiko}

		\item\raggedright \gl{mi leg·i·s}{קראתי} \gl{mi leg·a·s}{אני קורא} \gl{mi leg·o·s}{אקרא} \gl{leg·i}{לקרוא} \gl{leg·u}{קרא/י/ו!} \gl{mi leg·u·s}{הייתי קורא/ת (היפותטי)}

			\begin{gramatiko}
				\gramitem שלושה זמנים בסיסיים: עבר (\EO{-i-}), הווה (\EO{-a-}) ועתיד (\EO{-o-})~+ \EO{s}
				\gramitem אינפיניטיב (מקור): \EO{-i}
				\gramitem ציווי: \EO{-u}
				\gramitem ***: \EO{-us}
			\end{gramatiko}

		\item\raggedright \gl{mi leg·i·s la libr·o·n}{קראתי את הספר} \gl{la kat·o leg·a·s map·o·n}{החתול/ה קורא/ת מפה} \gl{li vid·o·s mi·n}{הוא יראה אותי} \gl{mi ven·i·s, mi ved·i·s, mi venk·i·s}{באתי, ראיתי, ניצחתי}
		\item\raggedright \gl{ŝi manĝ·i·s ŝi·a·n pan·o·n}{היא אכלה את הלחם שלה (של מישהי אחרת)} \gl{ŝi manĝ·i·s si·a·n pan·o·n}{היא אכלה את הלחם שלה (עצמה)}
		\item\raggedright \gl{la libr·o est·a·s interes·a}{הספר מעניין} \gl{la mamut·o·j est·i·s mam·ul·o·j}{הממותות היו יונקים} \gl{est·i·s dinosaŭr·o·j grand·a·j kaj mal·grand·a·j}{היו דינוזאורים גדולים וקטנים} \gl{est·a·s kat·o sur la arb·o}{יש חתול על העץ}
		\item\raggedright \gl{mi ne sci·a·s}{אני לא יודע/ת} \gl{ne mi manĝ·i·s la pom·o·n}{לא אני אכלתי את התפוח} \gl{ĉu est·i aŭ ne est·i?}{להיות או לא להיות?}
		\item\raggedright \gl{Nimbus·o 2000 est·a·s la plej rapid·a bala·il·o; ĝi flug·a·s rapid·e}{נימבוס 2000 הוא המטאטא המהיר ביותר; הוא עף מהר}
		\item\raggedright \gl{banan·o·j est·a·s pli bon·gust·a ol melongen·o·j}{בננות טעימות יותר מחצילים} \gl{mi skrib·a·s ti·el bon·e ki·el vi}{אני כותב/ת טוב כמוך}
		\item\raggedright \gl{la lud·i·nt·a infan·o}{הילד/ה ששיחק/ה} \gl{la lud·a·nt·a infan·o}{הילד/ה המשחק/ת} \gl{la lud·o·nt·a infan·o}{הילד/ה שת/ישחק} \gl{la infan·o est·a·s lud·a·nt·a}{הילד/ה משחק/ת} \gl{la infan·o est·a·s lud·i·nt·a}{הילד/ה כבר שיחק/ה} \gl{la infan·o est·a·s lud·o·nt·a}{הילד/ה עומד/ת לשחק} \gl{la infano·o est·i·s lud·i·nt·a}{הילדה שיחקה בעבר}
		\item\raggedright \gl{la lud·a·nt·o}{הַמְּשַׂחֵק} \gl{la esper·a·nt·o}{הַמְּקַוֶּה} \gl{kant·i·nt·e, ŝi paŝ·i·s}{בעודה שרה, היא הלכה}
		\item\raggedright \gl{manĝ·i·t·a pom·o}{תפוח אכוּל} \gl{la pom·o est·a·s manĝ·i·t·a}{התפוח אכוּל} \gl{mem·instru·i·t·o}{אוטודידקט/ית}
		\item\raggedright \gl{oni romp·i·s la fenestr·o·n}{שָׁבְרוּ את החלון} \gl{la fenestr·o est·a·s romp·i·t·a}{החלון שבור (מישהו שבר)} \gl{la fenestr·o romp·iĝ·i·s}{החלון נשבר (מעצמו)} \gl{la fenestr·o est·a·s romp·o·t·a de la pilk·o}{החלון עומד להשבר מהכדור}
		\item\raggedright \gl{ĉes·u bru·i, mi vol·a·s dorm·i!}{תפסיק/י/ו להרעיש, אני רוצה לישון!}
		\item\raggedright \gl{ĉu vi vol·a·s lud·i?}{רוצה לשחק?} \gl{vi est·a·s Ludviko, ĉu ne?}{אתה לודוויק, לא?} \gl{mi ne sci·a·s, ĉu li ven·o·s}{אני לא יודע/ת אם הוא יבוא} \gl{ĉu vi ne ir·i·s? —ne~/ jes~/ ĝuste~/ mal·ĝuste}{לא באת? —לא~/ כן~/ אכן~/ אכן לא}
		\item
			\rotatebox{-90}{\begin{tabular}{c|ccccc}
				& \gloseto{ki-}{שאלה} & \gloseto{ti-}{הצבעה} & \gloseto{i-}{סיתום} & \gloseto{ĉi-}{הכללה} & \gloseto{neni-}{שלילה}\\
				\hline
				\gloseto{-a}{תכונה} & \gloseto{ki·a}{איזה (סוג)} & \gloseto{ti·a}{כזה} & \gloseto{i·a}{איזשהו} & \gloseto{ĉi·a}{כל (סוג)} & \gloseto{neni·a}{אף (סוג)}\\
				\gloseto{-al}{סיבה} & \gloseto{ki·al}{למה} & \gloseto{ti·al}{בגלל זה} & \gloseto{i·al}{מאיזושהי סיבה} & \gloseto{ĉi·al}{מכל הסיבות} & \gloseto{neni·al}{בלי סיבה}\\
				\gloseto{-am}{זמן} & \gloseto{ki·am}{מתי} & \gloseto{ti·am}{אז} & \gloseto{i·am}{מתישהו} & \gloseto{ĉi·am}{תמיד} & \gloseto{neni·am}{אף פעם}\\
				\gloseto{-e}{מקום} & \gloseto{ki·e}{איפה} & \gloseto{ti·e}{שם} & \gloseto{i·e}{איפשהו} & \gloseto{ĉi·e}{בכל מקום} & \gloseto{neni·e}{בשום מקום}\\
				\gloseto{-el}{אופן} & \gloseto{ki·el}{איך} & \gloseto{ti·el}{כך} & \gloseto{i·el}{איכשהו} & \gloseto{ĉi·el}{בכל דרך} & \gloseto{neni·el}{בשום דרך}\\
				\gloseto{-es}{שייכות} & \gloseto{ki·es}{של מי} & \gloseto{ti·es}{של זה} & \gloseto{i·es}{של מישהו} & \gloseto{ĉi·es}{של כולם} & \gloseto{neni·es}{של אף אחד}\\
				\gloseto{-o}{דבר} & \gloseto{ki·o}{מה} & \gloseto{ti·o}{זה} & \gloseto{i·o}{משהו} & \gloseto{ĉi·o}{הכל} & \gloseto{neni·o}{שום דבר}\\
				\gloseto{-om}{כמות} & \gloseto{ki·om}{כמה} & \gloseto{ti·om}{כזאת כמות} & \gloseto{i·om}{קצת, איזושהי כמות} & \gloseto{ĉi·om}{הכל, כל הכמות} & \gloseto{neni·om}{כלום, שום כמות}\\
				\gloseto{-u}{מישהי/ו} & \gloseto{ki·u}{מי, איזה (אחד)} & \gloseto{ti·u}{הזה} & \gloseto{i·u}{מישהו} & \gloseto{ĉi·u}{כולם} & \gloseto{neni·u}{אף אחד}\\
		\end{tabular}}
	\end{enumerate}
\end{LTR}



\bisection{מבנה מילים}{{Vortfarado}}

%\newcommand{\afikso}[4]{\ducxirkauxi{\EO{#1}} & #2 & #3 & #4\\}
%\newcommand{\afikso}[4]{\centering \gl{\textbf{#1}}{#2} & #3 & #4\\}
\newcommand{\afikso}[4]{\parbox{0.2\textwidth}{\centering \gl{\textbf{#1}}{#2}} \parbox{0.5\textwidth}{\centering #3} \parbox{0.2\textwidth}{\centering #4}\\}
%\newcommand{\derivajxo}[4]{\gloseto{#1}{#2}~\resizebox{2ex}{!}{\EO{→}}~\gloseto{#3}{#4}}
\newcommand{\derivajxo}[4]{\begin{tabular}{p{9em}@{\resizebox{2ex}{!}{\EO{→}}}p{5em}}\centering\gloseto{#1}{#2}&\centering\gloseto{#3}{#4}\end{tabular}}

%\begin{tabulary}{\textwidth}{p{0.05\textwidth}p{0.10\textwidth}p{0.40\textwidth}p{0.25\textwidth}}
%\begin{tabulary}{\textwidth}{CCC}
%\begin{tabu} to \linewidth {cccc}

%\textbf{סיומות}:
\bisubsection{סיומות}{Postafiksoj}

\afikso{aĉ}
{גינוי}
{\derivajxo{skrib·\hl{aĉ}·i}{לקשקש}{skrib·i}{לכתוב}}
{\gl{aĵ·\hl{aĉ}·o}{זבל} \gl{aĉ!}{איכס!}}

\afikso{ad}
{חזרה~/ המשכיות}
{\derivajxo{kur·\hl{ad}·i}{לרוץ ולרוץ}{kur·i}{לרוץ}}
{\gl{\hl{ad}·i}{להמשיך} \gl{\hl{ad}·a}{המשכי}}

\afikso{aĵ}
{ביטוי מסויים, תוצר}
{\derivajxo{nov·\hl{aĵ}·o}{חדשות, חדשה}{nov·a}{חדש}}
{\gloseto{aĉ·ig·\hl{aĵ}·o}{ברדק (מסויים)}}

\afikso{an}
{חברות בקבוצה}
{\derivajxo{ŝip·\hl{an}·o}{אחד מצוות הספינה}{ŝip·o}{ספינה}\break \derivajxo{amerik·\hl{an}·o}{אמריקאי}{amerik·o}{אמריקה}}
{\gloseto{\hl{an}·o}{חבר (במשהו)}}

\afikso{ar}
{\pbox{100cm}{קבוצה (ללא\\מספר מסויים)}}
{\derivajxo{vort·\hl{ar}·o}{מילון}{vort·o}{מילה}\break\derivajxo{hom·\hl{ar}·o}{אנושות}{hom·o}{אדם}}
{\gl{an·\hl{ar}·o}{חֶבְרָה} \gloseto{\hl{ar}·o}{עדר, קבוצה}}

\afikso{*ĉj}
{חיבה (\symbolglyph{♂})}
{\derivajxo{pa\hl{ĉj}o}{אבאל׳ה}{patr·o}{הוֹרֶה}\\\derivajxo{Mi\hl{ĉj}o}{מיקי}{Miĥael·o}{מיכאל}}
{\gloseto{la i\hl{ĉj}·o·j}{„הבחורים”, „הבנים”}}

\afikso{ebl}
{יכולת}
{\derivajxo{vid·\hl{ebl}·a}{נראה}{vid·i}{לראות}}
{\gl{ebl·e}{אולי}}

\afikso{ec}
{מופשט}
{\derivajxo{amik·\hl{ec}·o}{ידידוּת}{amik·a}{ידידותי}}
{\gl{\hl{ec}·ar·o}{אופי}}

\afikso{eg}
{הגדלה}
{\derivajxo{vir·\hl{eg}·o}{ענק}{vir·o}{איש}}
{\gl{\hl{eg}·a}{עצום}}

\afikso{ej}
{מקום, ־ִיָּה}
{\derivajxo{vend·\hl{ej}·o}{חנות}{vend·i}{למכור}}
{\gl{\hl{ej}·o}{מקום}}

\afikso{el}
{שבח}
{\derivajxo{skrib·\hl{el}·o}{כתיבה תמה}{skrib·o}{כתיבה}}
{}

\afikso{em}
{נטיה}
{\derivajxo{lud·\hl{em}·a}{משחקי/ת}{lud·i}{לשחק}}
{\gl{\hl{em}·o}{נטיה} \gl{mal·\hl{em}·a}{ממאן}}

\afikso{end}
{חובה}
{\derivajxo{leg·\hl{end}·a}{שחייבים לקרוא}{leg·i}{לקרוא}}
{}

\afikso{er}
{מרכיב קטן}
{\derivajxo{ĉen·\hl{er}·o}{חוליה}{ĉen·o}{שרשרת}}
{\gl{\hl{er}·o}{פירור}}

\afikso{estr}
{ראש־}
{\derivajxo{urb·\hl{estr}·o}{ראש־עירייה}{urb·o}{עיר}}
{\gl{estr·\hl{ar}·o}{מועצת מנהלות/ים}}

\afikso{et}
{הקטנה}
{\derivajxo{libr·\hl{et}·o}{ספרון}{libr·o}{סֵפֶר}}
{\gl{\hl{et}·e}{בקושי}}

\afikso{i}
{ארץ־}
{\derivajxo{Angl·\hl{i}·o}{אנגליה}{angl·o}{אנגלי/ה}}
{}

\afikso{iĉ}
{זכר}
{\derivajxo{patr·\hl{iĉ}·o}{אבא}{patr·o}{הוֹרֶה}}
{}

\afikso{id}
{צאצא}
{\derivajxo{reĝ·\hl{id}·o}{נסיך/ה}{reĝ·o}{מלך/ה}}
{\gl{\hl{id}·o}{ילד, גור}}

\afikso{ig}
{גרימה}
{\derivajxo{mort·\hl{ig}·i}{להרוג}{mort·i}{למות}}
{\gl{\hl{ig}·i}{לגרום}}

\afikso{iĝ}
{(התפעל)}
{\derivajxo{amuz·\hl{iĝ}·i}{להשתעשע}{amuz·i}{לשעשע}}
{\gl{\hl{iĝ}·i}{להפוך ל־, להעשות}}

\afikso{il}
{מכשיר, כלי}
{\derivajxo{tranĉ·\hl{il}·o}{סכין}{tranĉ·i}{לפרוס}}
{\gl{\hl{il}·o}{מכשיר, כלי}}

\afikso{in}
{נקבה}
{\derivajxo{patr·\hl{in}·o}{אמא}{patr·o}{הוֹרֶה}}
{\gl{\hl{in}·o}{נקבה}}

\afikso{ind}
{ראוי ל־}
{\derivajxo{kred·\hl{ind}·a}{אמין/ה}{kred·i}{להאמין}}
{\gl{\hl{ind}·a}{ראוי/ה}}

\afikso{ing}
{מִכְסֶה, מחזיק}
{\derivajxo{glav·\hl{ing}·o}{נדן}{glav·o}{חרב}}
{\gl{\hl{ing}·o}{מכסה, תושבת}}

\afikso{ism}
{תפישה, ־יזְם}
{\derivajxo{komun·\hl{ism}·o}{קומוניזם}{komun·a}{משותף}}
{\gl{\hl{ism}·o}{אִיזְם}}

\afikso{ist}
{מקצוע~/ תומך}
{\derivajxo{instru·\hl{ist}·o}{מורה}{instru·i}{ללמד, להורות}}
{\gl{\hl{ist}·o}{מקצוען/ית}}

\afikso{*nj}
{חיבה (\symbolglyph{♀})}
{\derivajxo{pa\hl{nj}o}{אמאל׳ה}{patr·o}{הוֹרֶה}\\\derivajxo{Jo\hl{nj}o}{ג׳ואני}{Joan}{ג׳ואן}}
{\gl{la i\hl{nj}·o·j}{„הבנות”}}

\afikso{obl}
{כפולה}
{\derivajxo{tri·\hl{obl}·e}{פי שלוש}{tri}{3}}
{\gl{\hl{obl}·e}{יותר מפעם אחת}}

\afikso{on}
{שבר}
{\derivajxo{du·\hl{on}·o}{חצי}{du}{2}}
{\gl{\hl{on}·ô}{שבר}}

\afikso{op}
{\pbox{100cm}{קבוצה (עם\\מספר מסויים)}}
{\derivajxo{tri·\hl{op}·o}{טרילוגיה}{tri}{3}}
{\gl{\hl{op}·o}{קבוצה, צוות}}

\afikso{uj}
{קיבול}
{\derivajxo{mon·\hl{uj}·o}{ארנק}{mon·o}{כסף}}
{\gl{\hl{uj}·o}{מיכל}}

\afikso{ul}
{מישהו}
{\derivajxo{aboco·\hl{ul}·o}{לומד/ת קריאה}{aboco}{א״ב}\\\derivajxo{proksim·\hl{ul}·o}{שכן}{proksim·a}{קרוב}}
{\gl{ti·am·\hl{ul}·o}{בת/בן אותו הזמן}}

\afikso{um}
{\pbox{100cm}{(מנעד רחב\\של משמעויות)}}
{\derivajxo{aer·\hl{um}·i}{לאוורר}{aer·o}{אוויר}}
{\gl{\hl{um}·o}{מה־שמו}}

%\EO{sam·ide·an·o}
%\EO{krokodil·i}
%\EO{volapuk·aĵ·o}
%\EO{de·nask·ul·o}
%\EO{rideti ridegi}
%\EO{id·ar·o}
%\EO{maltrinki / malmanĝi}
%קללות



\bisubsection{תחיליות}{Antauxafiksoj}

\afikso{bo}
{מחותנוּת (\L{\textasciitilde~-in-law})}
{\derivajxo{\hl{bo}·patr·in·o}{חותנת}{patr·in·o}{אמא}}
{}

\afikso{dis}
{פירוד}
{\derivajxo{\hl{dis}·ĵet·i}{להעיף לכל עבר}{ĵet·i}{לזרוק}}
{\gl{\hl{dis}·ig·i}{להפריד} \gl{\hl{dis}·iĝ·i}{להפרד} \gl{\hl{dis}!}{תְ׳חפף!}}

\afikso{ek}
{תחיליוּת, פתאומיוּת}
{\derivajxo{\hl{ek}·am·i}{להתאהב}{am·i}{לאהוב}}
{\gl{\hl{ek}·i}{להתחיל} \gl{\hl{ek}·de}{מאז־} \gl{\hl{ek}—!}{הלאה־!}}

\afikso{eks}
{לשעבר, אֶקְס־}
{\derivajxo{\hl{eks}·edz·iĝ·i}{להתגרש}{edz·iĝ·i}{להתחתן}}
{\gl{\hl{eks}·a}{קוֹדֵם} \gl{\hl{eks} la reĝo!}{{\textasciitilde}~לעזאזל עם המלך!}}

\afikso{fi}
{גועל, בושה}
{\derivajxo{\hl{fi}·vort·o}{ניבול־פה}{vort·o}{מילה}}
{\gl{\hl{fi}·a}{נאלח} \gl{\hl{fi} al vi!}{תתבייש!}}

\afikso{ge}
{רב־מגדרי~/ ־מיני}
{\derivajxo{\hl{ge}·sinjor·o·j}{גבירותי ורבותי}{sinjor·o}{גברת~/ אדון} \derivajxo{\hl{ge}·lern·ej·o}{בית־ספר מעורב מגדרית}{lern·ej·o}{בית־ספר}}
{\gl{\hl{ge}·iĝ·i}{להזדווג} \gl{\hl{ge}·a}{הטרוסקסואלי}}

\afikso{mal}
{היפוך, ניגוד}
{\derivajxo{\hl{mal}·grand·a}{קטן}{grand·a}{גדול} \derivajxo{\hl{mal}·obe·i}{לסרב (לציית)}{obe·i}{לציית}}
{\gl{\hl{mal}·a}{הפוך, מנוגד}}

\afikso{mis}
{באופן שגוי~/ משובש}
{\derivajxo{\hl{mis}·lok·i}{לשים במקום שגוי}{lok·i}{למקם}}
{\gl{\hl{mis}·e}{באופן שגוי}}

\afikso{pra}
{קדום, קדמוני}
{\derivajxo{\hl{pra}·av·o}{סב(ת)א רב(ת)א}{av·o}{סב(ת)א}}
{\gl{\hl{pra}·a}{קדום, קדמוני}}

\afikso{re}
{שוב, מחדש}
{\derivajxo{\hl{re}·konstru·i}{לבנות מחדש}{konstru·i}{לבנות} \derivajxo{\hl{re}·bril·o}{השתקפות}{bril·o}{ברק, בוהַק}}
{}
